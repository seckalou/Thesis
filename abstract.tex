We introduce new methods of analysing three dimensional surface texture on high resolution normal fields and apply these to detect and assess skin conditions on human face, specifically wrinkle, pore and acne. 

Computer aided skin condition assessment has been mostly addressed using two dimensional texture analysis techniques on skin images or coarse geometrical features extracted from the skin three dimensional macro structures. While the first trend ignores the three dimensional nature characterising most of the skin conditions, the latter mainly deal with geometrical features that are not fine enough to capture the skin structures in the meso and micro scales.

However, the recent advances in three dimensional surface imaging during the last decades brings the possibility of capturing human skin fine geometrical structures and reflectance properties with unprecedented quality and resolution (down to the level of the pores). The methods proposed in this work aim at leveraging these advances and revisit the formulation of texture analysis as a three dimensional problem.

For data collection we set up a Lighstage to capture high resolution facial normal fields along with reflectance properties. The collected data are photo-realistically rendered and presented to experts for annotations according to presence of the studied skin conditions. This constitute the ground truth on which we  apply the proposed methods and learn models for detecting and assessing facial skin conditions.

We demonstrate as well that some of these three dimensional surface texture descriptors can be extended to synthesize highly detailed skin structures and simulate the studied skin condition on normal faces.